%%%%%%%%%%%%%%%%%%%%%%%%%%%%%%%%%%%%%%%%%
% Medium Length Professional CV
% LaTeX Template
% Version 2.0 (8/5/13)
%
% This template has been downloaded from:
% http://www.LaTeXTemplates.com
%
% Original author:
% Trey Hunner (http://www.treyhunner.com/)
%
% Important note:
% This template requires the resume.cls file to be in the same directory as the
% .tex file. The resume.cls file provides the resume style used for structuring the
% document.
%
%%%%%%%%%%%%%%%%%%%%%%%%%%%%%%%%%%%%%%%%%

%----------------------------------------------------------------------------------------
%	PACKAGES AND OTHER DOCUMENT CONFIGURATIONS
%----------------------------------------------------------------------------------------

\documentclass{resume} % Use the custom resume.cls style

\usepackage[left=0.75in,top=0.6in,right=0.75in,bottom=0.6in]{geometry} % Document margins
\usepackage{url,enumitem, xcolor}

\name{Peter Rindal} % Your name
\address{1445 NW Vista Pl. \\ Corvallis, OR 97330} % Your address
\address{\url{web.engr.OregonState.edu/~rindalp}} % Your secondary addess (optional)
\address{(509)~$\cdot$~520~$\cdot$~8701 \\ rindalp@OregonState.edu} % Your phone number and email

\begin{document}

%----------------------------------------------------------------------------------------
%	EDUCATION SECTION
%----------------------------------------------------------------------------------------

\begin{rSection}{Education}

{\bf Ph.D. in Computer Science} \hfill {\em January 2015 --- Est. June 2019} \\ 
Oregon State University, Corvallis\smallskip \\
Overall GPA: 3.9\\ 

{\bf B.S. in Computer Science} \hfill {\em September 2010 --- June 2014} \\ 
Oregon State University, Corvallis\smallskip \\
Overall GPA: 3.65\\ 

\end{rSection}


%----------------------------------------------------------------------------------------
%	Research Interests
%----------------------------------------------------------------------------------------

\begin{rSection}{Research Interests}
	
My primary interest is the development of efficient methods for computing on encrypted data. Most notably has been the development of a highly optimized protocol for performing general secure computation. I have also worked on Private Set Intersection for both malicious \& semi-honest adversaries, and several projects combining machine learning, differential privacy and secure computation.
	
\end{rSection}

%----------------------------------------------------------------------------------------
%	WORK EXPERIENCE SECTION
%----------------------------------------------------------------------------------------

\begin{rSection}{Employment}

{\bf Oregon State University} \hfill {January 2015 --- present}\\
{\emph{Graduate Research Assistant} \hfill {Corvallis, OR}}

{\bf Visa Research} \hfill {June 2017 --- September 2017}\\
{\emph{Security Research Intern} \hfill {Palo Alto, CA}}

{\bf Microsoft Research} \hfill {June 2016 --- September 2016}\\
{\emph{Security Research Intern} \hfill {Redmond, WA}}

{\bf Microsoft Research} \hfill {January 2016 --- March 2016}\\
{\emph{Security Research Intern} \hfill {Redmond, WA}}

{\bf Digimarc} \hfill {June 2014 --- December 2014}\\
{\emph{Software Developer Intern} \hfill {Portland, OR}}

{\bf Boeing Company} \hfill {March 2013 --- September 2013}\\
{\emph{Software Developer Intern} \hfill {Portland, OR}}

%\begin{rSubsection}{Visa Research}{June 2017 - September 2017}{Security Research Intern}{Palo Alto, CA}
%	\item Developed efficient protocols for training machine learning models on encrypted data with security against semi-honest and malicious adversaries.
%	\item Proposed a new definitions and constructions for distributed CPA, CCA and Authenticated Encryption where the key is securely split amongst several parties.
%\end{rSubsection}

%------------------------------------------------

\end{rSection}

%----------------------------------------------------------------------------------------
%	Publications
%----------------------------------------------------------------------------------------

\begin{rSection}{Publications}
\hfill {\scriptsize \textcolor{gray}{\emph{Note: the standard convention in this discipline is to list authors alphabetically.}}}

Peer-reviewed conference publications:
\begin{enumerate}[label=C\arabic*]
	
	\item Peter Rindal and Mike Rosulek. \emph{Faster Malicious 2-party Secure Computation with Online/Offline Dual Execution.} In \emph{USENIX Security Symposium 2016.}
	
	\item Gizem Cetin, Hao Chen, Kim Laine, Kristin Lauter, Peter Rindal and Yuhou Xia. \emph{Private Queries on Encrypted Genomic Data.} In \emph{BMC Medical Genomics:  iDASH Privacy and Security Workshop 2016.}
		
	\item Peter Rindal and Mike Rosulek. \emph{Improved Private Set Intersection against Malicious Adversaries.} In \emph{EUROCRYPT: International Cryptology Conference 2017.}
	
	\item Hao Chen, Kim Laine and Peter Rindal. \emph{Fast Private Set Intersection from Homomorphic Encryption.} In \emph{CCS: ACM Conference on Computer and Communications Security 2017.}
	
	\item Peter Rindal and Mike Rosulek. \emph{Malicious-Secure Private Set Intersection via Dual Execution.} In \emph{CCS: ACM Conference on Computer and Communications Security 2017.}
\end{enumerate}

\bigskip 

Informal publications:
\begin{enumerate}[label=I\arabic*]
	
	\item Ran Gilad-Bachrach, Kim Laine, Kristin Lauter, Peter Rindal and Mike Rosulek. \emph{Secure Data Exchange: A Marketplace in the Cloud.} In \emph{IACR ePrint 2016.}
		
	\item Peter Rindal and Roberto Trifiletti. \emph{SplitCommit: Implementing and Analyzing Homomorphic UC Commitments.} In \emph{IACR ePrint 2017.}
	
	\item Melissa Chase, Ran Gilad-Bachrach, Kim Laine, Kristin Lauter and Peter Rindal. \emph{Private Collaborative Neural Network Learning.} In \emph{IACR ePrint 2017.}
\end{enumerate}
 

\end{rSection}



%----------------------------------------------------------------------------------------
%	Presentations
%----------------------------------------------------------------------------------------

\begin{rSection}{Presentations}
	
	Conference and workshop presentations:
	\begin{enumerate}[label=P\arabic*]
		
		\item \emph{Faster Malicious 2-party Secure Computation with Online/Offline Dual Execution.} Usenix Security 2016, Austin Texas, USA, August 2016.
		
		\item \emph{Improved Private Set Intersection against Malicious Adversaries.} 
		\begin{itemize}
			\item Eurocrypt, Paris, France, April 2017.
			\item Theory and Practice of Secure Multiparty Computation, Bristol UK, April 2017.
		\end{itemize}
	\end{enumerate}
	
	\bigskip
	
	Other invited talks:	
	\begin{enumerate}[label=T\arabic*]
		
		\item \emph{A Survey of Oblivious RAM Methods and Optimizations.} Intel seminar, Hillsboro OR, USA, March 2015.
		
	\end{enumerate}
	
\end{rSection}


%----------------------------------------------------------------------------------------
%	Software Projects
%----------------------------------------------------------------------------------------

\begin{rSection}{Software Projects}
	
	
	\begin{enumerate}[label=S\arabic*]
		
		\item Peter Rindal. \emph{libOTe: A fast, portable, and easy to use Oblivious Transfer Library.} 
		
		\item Peter Rindal. \emph{Ivory-Runtime: A generic Secure Computation API for garbled circuits, SPDZ, etc. } 
		
		\item Peter Rindal and Ni Ni Triue. \emph{libPSI: A library for malicious and semi-honest Private Set Intersection (PSI).}
		
		\item Peter Rindal and Roberto Trifiletti. \emph{SplitCommit: A portable C++ implementation of the [FJNT16] XOR-homomorphic commitment scheme.}
		
		\item Peter Rindal. \emph{Batch Dual Execution: Malicious secure online/offline MPC implementation.} 
	\end{enumerate}
	
\end{rSection}



%----------------------------------------------------------------------------------------
%	Service
%----------------------------------------------------------------------------------------

\begin{rSection}{Service}
	
	External reviewer:
	\begin{enumerate}[label=E\arabic*]
		
		\item \emph{15th Theory of Cryptography Conference (TCC 2017).}  Baltimore, MD, USA on November, 2017.
		
		\item \emph{2nd IEEE European Symposium on Security and Privacy (EuroS\&P 2017).} Paris, France on April, 2017.
		
		\item \emph{19th International Symposium on Stabilization, Safety, and Security of Distributed Systems (SSS 2017).} Boston, Massachusetts, USA on November, 2017.
		
	\end{enumerate}
	
\end{rSection}



%----------------------------------------------------------------------------------------
%	Service
%----------------------------------------------------------------------------------------

\begin{rSection}{References}
	
	\begin{enumerate}[label=R\arabic*]
		
		\item Mike Rosulek, \emph{Principle Ph.D. Advisor.} rosulekm@eecs.oregonstate.edu
		
		\item Payman Mohassel,  \emph{Visa Research Mentor.} pmohasse@visa.com
		
		\item Melissa Chase, \emph{Microsoft Research Mentor.} melissac@microsoft.com
		
		
	\end{enumerate}
	
\end{rSection}


%----------------------------------------------------------------------------------------
%	EXAMPLE SECTION
%----------------------------------------------------------------------------------------

%\begin{rSection}{Section Name}

%Section content\ldots

%\end{rSection}

%----------------------------------------------------------------------------------------

\end{document}
